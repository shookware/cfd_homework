\documentclass[a4paper]{article}
\usepackage{amsmath}
\usepackage{geometry}
\geometry{left=2.5cm,right=2.5cm,top=4.5cm,bottom=4.5cm}
\begin{document}
\fontsize{12pt}{20pt}\selectfont
Considering the equation
\begin{equation}
  \frac{\partial u}{\partial t}+c\frac{\partial u}{\partial x}=0
\end{equation}

Here, $u=u(x,t)$ is the unkown function; $c$ is the convectional velocity. It determines the velocity of the wave broadcasting along the $x$ direction.
The boundary condition is:
\begin{equation}
  u(0, t)=1; u(L_x, t)=0.
\end{equation}

$0$ is the left boundary condition and $L_x$ is the right one.
The initial condition is:
\begin{equation}
  \begin{split}
  &u(x, 0)=1 \qquad   x\le L_x/2\\
  &u(x, t)=0 \qquad   x\ge L_x/2.
  \end{split}
\end{equation}

At first, we write the equation to a standard form such as:
\begin{equation}
  \frac{\partial u}{\partial t}=-c\frac{\partial u}{\partial x}
\end{equation}
Obviously, the right hand is the spatial terms of the equation and the left hand is the temporal variation. Here, we call the right hand ``RHS''.

Now, we use the 1-order scheme in both spatial and temporal direction.
After the explicit discretization, the equation becomes :
\begin{equation}
\frac{u^{n+1}_i-u^n_i}{\Delta t}=-c\frac{u_i^n-u_{i-1}^n}{\Delta x}=RHS
\end{equation}
here, the superscript dominates the temporal discretization and the subscript dominates the spatial one. $\Delta t$ is the time step and the $\Delta x$ is the spatial mesh scale.

 We let the $n-1$ terms to the left hand and the $n$ terms to the right hand. we obtained:
\begin{equation}
u^{n+1}_i=u^n_i+\Delta t RHS
\end{equation}
And then, the next time $u(x, t)$ can be obtained by this formulation.

However, the time step $\Delta t$ must satisfy some extra condition for the numerical statility. The condition we call it ``CFL condition''. It will be introducted in the following discussion. Here, the CFL condition is:
\begin{equation}
  \Delta t\le \Delta x/c
\end{equation}
we often use the following form:
\begin{equation}
  \Delta t=C_{cfl}\Delta x/c
\end{equation}
the $C_{cfl}$ is ``the Courant Number''. For the explicit scheme, $C_{cfl}\le 1$.
\newline
\textbf{Homework}\\
1. Programme and Compute $u(x,t)$. Draw the function in the tecplot.\\
2. Discuss the relationship among the parameter: $\Delta t, \Delta x$ and $c$. Find What happens when the CFL condition is not satified.\\
3. Try to use the central scheme in the spatial direction and find that if it is stable. \\
4. Modify the Programme, solve the equation \\
\begin{equation}
\frac {\partial u}{\partial t}+c\frac {\partial u}{\partial x}=\mu \frac {\partial^2 u}{\partial x^2}
\end{equation}
\end{document}
